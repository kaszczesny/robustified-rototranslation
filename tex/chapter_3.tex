\chapter{Discussion of results}

% global results for various datasets


\begin{lstlisting}
src
\end{lstlisting}

Pakiet \texttt{siunitx} zadba o to, by liczba została poprawnie sformatowana: \\
\begin{center}
	\num{1234567890.0987654321}
\end{center}


Pakiet \texttt{subcaption} pozwala na umieszczanie w podpisie rysunku odnośników do ,,podilustracji'': \\

\begin{figure}[h]
	\centering
	\begin{subfigure}{0.35\textwidth}
		\centering
		\framebox[2.0\width]{A}
		\subcaption{\label{subfigure_a}}
	\end{subfigure}
	\begin{subfigure}{0.35\textwidth}
		\centering
		\framebox[2.0\width]{B}
		\subcaption{\label{subfigure_b}}
	\end{subfigure}
	
	\caption{\label{fig:subcaption_example}Przykład użycia \texttt{\textbackslash subcaption}: \protect\subref{subfigure_a} litera A, \protect\subref{subfigure_b} litera B.}
\end{figure}

