
\addcontentsline{toc}{chapter}{Abstract}
\chapter*{Abstract}


%Uzasadnienie podjęcia tematu – musimy odpowiedzieć na pytanie, jak i dlaczego ważne jest to, o czym napisaliśmy.
Egomotion is an underlying problem behind SLAM systems and in some cases can be more accurate than GPS systems.

%Problem badawczy – napiszmy jasno, jaki był cel badania, co badaliśmy, jaki problem intelektualny/praktyczny staraliśmy się rozwiązać.
Problem present in most GPS applications is accuracy of the readings while making sharp turns, for example on roundabouts, which may lead to taking the wrong route. To increase this accuracy, Visual Odometry could be used. Rebvo algorithm was tested for that routine, as creators of said algorithm point out its real-time capabilities.

%Metodologia – należy opisać jak staraliśmy się odpowiedzieć na pytanie z poprzedniego punktu. Perspektywa, metody, techniki.
Rebvo algorithm was implemented in Octave/OpenCV/mex to test how it behaves under specific conditions and whether it will suit aforementioned environment.

%Wyniki – streszczamy to, co nasze badanie wykazało.
%Implikacje – abstrakt zamykamy wnioskami, które płyną z naszej pracy.
Tests show convergence with algorithm authors' publication, and conclude that said algorithm would work well under conditions mentioned before, making it useful in implementation of GPS units' performance increasing system.

%Słowa kluczowe – dla porządku i ułatwienia sklasyfikowania naszej pracy opisujemy ją słowami kluczowymi.
Keywords: Visual odometry, egomotion, motion estimation, SLAM

\addcontentsline{toc}{chapter}{Introduction}
\chapter*{Introduction}


Odometry is a term describing measurement of distance (from Greek: \textit{odos}~\==~ ``path``, \textit{metron}~\==~``measure``). For instance, car odometer is a device that calculates total traveled distance through multiplication of wheel diameter by the number of wheel spins. Visual odometry aims to determine the \textit{egomotion} of an object -- its position relative to a rigid scene -- by observing changes in video registered by one or more cameras attached to said object.

% to proste i tanie rozwiazanie dla lokalizacji kamery w porownaniu do gpsow, zyroskopow etc, a do tego kazdy ma durzo kamer bo sa wszedzie. jest wykorzystywana do stabilizacji obrazu, augmented reality albo po prostu w systemach SLAM(Self LOcalization And Mapping) wykorzystywanego w robotach/dronach do pomocy w autonomicznym poruszaniu sie czegos takiego lub mapowania pomieszczen. w tym paperze zajmujemy sie jednak tylko pozycja kamery ale nie jako system slam, bo nie trzymamy historii.
% POKEMON GO TO THE MAGISTERKA

Main advantage of visual approach to odometry is ubiquity of cameras embedded into virtually every mobile device -- be it a smarphone, game console or even a smartwatch. Not every of those devices has a gyroscope, and even if a GPS (Global Positioning System) chip is present, its locational accuracy is low. Practical applications of visual odometry include:
\begin{itemize}
	\item SLAM (Self\=/Localization And Mapping) used by autonomous robots, e.g. Mars Exploration Rovers, drones or self-driving cars.
	
	\item Handheld video stabilization, realized in software.
	
	\item AR (Augmented Reality) -- last year in particular saw the emergence of \textit{Pok\'emon`GO} \cite{pineco}, an acclaimed mobile game. Recently Google has released \textit{ARCore}~\cite{androidvr} , a software development kit, który jest przeznaczony na telefony obficie czerpie z rozwiązań wizualnej odometrii.
\end{itemize}


Following this trend of mobile cośtam, a vision-based driving assistance system was envisioned ktory by korzystal tylko z tego co ma kazdy tel: kamera i gps, oczywiscie dzialajac real time. zastosowaniem mialo byc wspieranie gps ....


%Full SLAM systems are often computationally demanding and are best suited for dedicated devices, which smartphones are not. Instead we looked for a lightweight solution for increasing accuracy of GPS data for better navigation, for example on roundabouts.



%This paper focuses on the latter - self localization is particularly important on autonomous devices, such as drones, robots or even self-driving cars (Mars Rover used that technology /jak to napisac ladnie/). In this case, algorithm is not a full SLAM one - no global map or keypoint history are generated.
% byl sobie landmark paper, czeste podejscie to slam ale niewydajny bo i7, slam w Roverze, a mysmy chcieli cos embedded na telefony komorkowe w normalnej cenie a nie z zyroskopem. strona odjarka z androidem zacytowac (?) moze cos o state of the art?, 




%nasz paper zawiera analize zmodyfkowanej wersji REBWO, nie jest na telefonie bo jestesmy slabiaki (cieniasy?), nasza implementacja to tylko prototyp bo mialo byc na telefony ale nie jest szypkie ale tez bylo niby realtime wiec to bylby rozwoj programu, analiza tego algorytmu obejmuje wszystkie kroki po akwizycji obrazu bo po chuj nam drona a z drugiej strony chcemy zaczac od obrazu stricte
We chose Rebvo algorithm, created by Tarrio and Pedre, since they presented a quick, accurate, and most importantly unique algorithm for visual odometry. This paper contains detailed analysis of Rebvo algorithm which is a prototype for smartphone application. Their implementation is supposed to be real-time on low-end processors (ARM).
Each step of the algorithm flow from image capture was analyzed and tested on video sequences from publicly available datasets containing ground truth for validating results. Our results were comparable to those achieved from state-of-the-art algorithms, which ensures that this implementation would make a good basis for aforementioned application.
%podsumujemy sobie ze testowalismy a sekwencjach z dwoch datasetow a wyniki sa spoko jak sa spoko, bo jak sie zepsuje to nie XD

This thesis is organized as follows:

PODZIAL PRACY
 edgefinder jointly, joiner ja, minimizer ty, matcher ja, kalman ty :S



\clearpage 