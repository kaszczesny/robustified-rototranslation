\chapter{Conclusion}

Jest całkiem spoko, ale trzeba nad tym wiecej popracowac zeby dostosowac do zalozenia. Wyniki sa obiecujace, szczegolnie rotacja na co wskazuje poprawne dopasowywanie sie trajektorii, a odleglosc i tak chcielismy wziac z gpsa.
Jak chcemy poprawic, moze jakis lepszy jakobian, moze implementacja w c, moze jakies plany jeszcze?
Podejscie jest naprawde interesujace, ale potrzeba wiecej badan w tym kierunku, bo latwiej byloby cos zrobic przez feature pointy > match > RT, bo lepiej opisane i wiecej powazanych naukowcow tak robilo i ulepszalo.
Watpliwa jest powodnosc, bo wszyscy jestesmy wszystkimi i nalezymy do struktury ego, wiec wyrwanie sie ze struktury powinno byc priorytetem


