\chapter{Pierwszy dokument}
\label{cha:pierwszyDokument}

W rozdziale tym przedstawiono podstawowe informacje dotyczące struktury prostych plików \LaTeX a. Omówiono również metody kompilacji plików z zastosowaniem programów \emph{latex} oraz \emph{pdflatex}.



\subsection{Notation (Keyline structure)}

Pixels that contain subpixel edge positions are called Keylines and, after edge extraction, are stored as an array of structures defined in Table \ref{tab:keyline}.

\begin{table}[h]
	\centering
	
	\begin{threeparttable}
		\caption{Keyline structure}
		\label{tab:keyline}
		
		\begin{tabularx}{0.6\textwidth}{C{1} C{1}}
			\toprule
			\thead{Structure field\tnote{a}} & \thead{Description} \\
			\midrule
			$q$ & subpixel position in image \\
			$h$ & normalized $q$ (principal point moved to $(0,0)$ and coordinates divided by focal length $z_f$ \\
			$\rho$, $\sigma_{\rho}$ & estimated inverse depth: $\frac{1}{Z}$, and its variance \\
			$\rho_0$, $\sigma_{\rho_{0}}$ & inverse depth predicted by Kalman filter and its variance \\
			$g$ & edge gradient obtained from DoG \\
			$p_{id}$ & index of previous Keyline in an edge \\
			$n_{id}$ & index of next Keyline in an edge \\
			
			matching/forward/historia \\
			$h_0$ &  \\
			$g_{0}$ & gradient of matched Keyline from previous frame \\
			\bottomrule
		\end{tabularx}
		
		\begin{tablenotes}
			\footnotesize
			\item[a] Jakiś komentarz\textellipsis
		\end{tablenotes}
		
	\end{threeparttable}
\end{table}

% ---

\subsection{Edge extraction}

First main step of algorithm is edge extraction with subpixel accuracy. Extracted Keylines that are estimated to lie on same edge are joined.

\subsubsection{Zero crossing of DoG}
While many edge detection algorithms could be used in this step, authors of \textit{Rebvo} have chosen Difference of Gaussians, because it provides \cite{jose2015realtime}:
\begin{enumerate}
	\item repetivity -- an edge is detected similarly throughout consecutive frames,
	\item precision -- edge positions are accurate,
	\item low time \& memory complexity.
\end{enumerate}

Another advantage of DoG is that edge gradient can be obtained directly from normal vector of the fitted local plane.

\subsubsection{Keyline joining}

After obtaining individual Keylines, they are joined together to form continuous (from consecutive pixels) edges. % todo footnote
For each Keyline, its neighbors are searched among 8 bordering pixels in direction perpendicular to edge gradient.
Most of the time, the algorithm considers only individual Keylines. Their neighbors are used only in Matching step to ... % todo
and in Regularization step, when ... % todo
is averaged over Keyline and its neighbors.


%--

\subsection{Edge tracking}

During minimization, previous and current Keylines are initially@ matched

\subsubsection{Warping function}

\subsubsection{Auxiliary image}

\subsubsection{Keyline matching criteria}

% todo Jan: figure

\subsubsection{Energy minimization}

\subsubsection{Initial conditions}

\subsubsection{Reweighing}

%--

\subsection{Mapping}

\subsubsection{Forward matching}

\subsubsection{Directed matching}

\subsubsection{Regularization}

\subsubsection{Depth estimation}

\subsubsection{Scale correction}



% ----------------------
% frame for rectification leftovers removal
% DOG is performed + example
%for every px minus windows:
% test 1: check if gradient if sufficient (todo check if most px are discarded)
% test2: check if plues and minues are balanced, 1px edge
% fitting to plane
% test 2.9: check if fitting was ok
% test 3: zero corssing; if lies inside the pixel
% test on dog: check if edge gradient is strong enough
% image: quiver
% todo: prove that noise for rho is better than ones

% img_mask with KL indeces

% edge joining + pruning
% for every keyline:
%   get direction perpendicular to gradient
%   search in this direction among immediate neighbors in image - if such neighbor is a KL, is belongs to the same edge
% image: joined edges
% pruning: edge ends and 3-px edges are removed



% minimizer
% auxiliary image: a lookup table for minimizer
% estimate quantile - its purpose is to cutoff KLs that have too high uncertainty with respect to all KLs
% project KL_prev to 3D using previously estimated rhos
% one iteration of LM:
%     using jacobian, new state vector is calculated for next iteration

%     apply transformation given in argument
%     project previous points from these 3D positions to 2D
%     for every previous KL perfoms test to check if it should take part in minimization:
%        test1: uncertainty below previously estimated quantile
%		 test2: (excluding minimalization for very first frame) check if KL has appeared before
%	     estimate rewieghting
%		 test3: check if projected KL lies within image frame
%        test4: check if there is a KL in this pixel using auxiliary image
%        test5: compare their gradients (escobar)
%        we get residual (DResidual) & weighted residual (fm) projected in direction of gradient
%      calculate jacobians using fishy equations
%      if there was gain, use these new parameters, otherwise ...
% double init(zeros and priors): 3 iterations with no reweighting
% use better result, proceed for 15 iterations with reweighting
% transformation that resulted in lowest score is considered the optimal transformation between frames
% estimate uncertainty as inverse of final jacobian

% forward rotate
% KL pixel positions, represented in normalized homo coords, are rotated using obtained rotation
% if they don't go to infinity, they are projected back onto image plane
% & nasty gradient rotation

% forward match
% using information about which KLs were matched to which, following fields are propagated: rho, s_rho, gradient, history and index of the matched KL
% todo: could new edges apprer without directed match

%directed match
% for each KL:
%   backrotate KL and cast it onto old edge map mask
%   using this position and backrotated velocity, obtain halfline in which this KL could have been moved (a halfline, because we only know that its rho is positive) - estimated pixel displacement
%   estimate uncertainty using projected position and Rvel
%   caonstrain displacement search radius & define starting point
% todo: check if perpedicular displacement if really common
%
% search from starting point alternating between one direction and the other
% check if this pixel there is a KL to match to
% perform 3 tests to filter outliers + image (eg. 3 consecutive images where an outlier disappears):
%  gradient angle similarity
%  gradient modulus similarity
%  motion consistency

% if there aren't enough matched KLs (500), reset

% optional regularization, performed twice
% main assumption is that KLs neighboring on an edge ale located near each other in 3D, so their depth should be similiar
% image: show that this isn't ALWAYS true
% for each KL:
%  check if 2 neighbors pass test:
%   if depths outweight uncertainties (probabilistic uncertainty)
%   if angle between gradients is below threshold
%  if tests are passed, rho and s_rho are smoothened taking weighted mean of its value and its neighbors

% kalman fiter
% kill me pls
% inverse depth is constrained between certain min and max

% optional estimate rescaling
% according to Tarrio et al, EKF is biased in rho estimation, so a global "shrinking factor" can be applied to depths and unvertainties.


%---------------------------------------------------------------------------



\section{Struktura dokumentu}
\label{sec:strukturaDokumentu}

Plik \LaTeX owy jest plikiem tekstowym, który oprócz tekstu zawiera polecenia formatujące ten tekst (analogicznie do języka HTML). Plik składa się z dwóch części:
\begin{enumerate}%[1)]
\item Preambuły -- określającej klasę dokumentu oraz zawierającej m.in. polecenia dołączającej dodatkowe pakiety;

\item Części głównej -- zawierającej zasadniczą treść dokumentu.
\end{enumerate}


\begin{lstlisting}
\documentclass[a4paper,12pt]{article}      % preambuła
\usepackage[polish]{babel}
\usepackage[utf8]{inputenc}
\usepackage[T1]{fontenc}
\usepackage{times}

\begin{document}                           % część główna

\section{Sztuczne życie}

% treść
% ąśężźćńłóĘŚĄŻŹĆŃÓŁ

\end{document}
\end{lstlisting}

Nie ma żadnych przeciwskazań do tworzenia dokumentów w~\LaTeX u w~języku polskim. Plik źródłowy jest zwykłym plikiem tekstowym i~do jego przygotowania można użyć dowolnego edytora tekstów, a~polskie znaki wprowadzać używając prawego klawisza \texttt{Alt}. Jeżeli po kompilacji dokumentu polskie znaki nie są wyświetlane poprawnie, to na 95\% źle określono sposób kodowania znaków (należy zmienić opcje wykorzystywanych pakietów).


%---------------------------------------------------------------------------

\section{Kompilacja}
\label{sec:kompilacja}


Załóżmy, że przygotowany przez nas dokument zapisany jest w pliku \texttt{test.tex}. Kolejno wykonane poniższe polecenia (pod warunkiem, że w pierwszym przypadku nie wykryto błędów i kompilacja zakończyła się sukcesem) pozwalają uzyskać nasz dokument w formacie pdf:
\begin{lstlisting}
latex test.tex
dvips test.dvi -o test.ps
ps2pdf test.ps
\end{lstlisting}
%
lub za pomocą PDF\LaTeX:
\begin{lstlisting}
pdflatex test.tex
\end{lstlisting}

Przy pierwszej kompilacji po zmiane tekstu, dodaniu nowych etykiet itp., \LaTeX~tworzy sobie spis rozdziałów, obrazków, tabel itp., a dopiero przy następnej kompilacji korzysta z tych informacji.

W pierwszym przypadku rysunki powinny być przygotowane w~formacie eps, a~w~drugim w~formacie pdf. Ponadto, jeżeli używamy polecenia \texttt{pdflatex test.tex} można wstawiać grafikę bitową (np. w formacie jpg).



%---------------------------------------------------------------------------

\section{Narzędzia}
\label{sec:narzedzia}


Do przygotowania pliku źródłowego może zostać wykorzystany dowolny edytor tekstowy. Niektóre edytory, np. GEdit, mają wbudowane moduły ułatwiające składanie tekstów w LaTeXu (kolorowanie składni, skrypty kompilacji, itp.).

Jednym z bardziej znanych środowisk do składania dokumentów  \LaTeX a jest {\em TeXstudio}, oferujące kompletne środowisko pracy. Zobacz: \url{http://www.texstudio.org}


Bardzo dobrym środowiskiem jest również edytor gEdit z wtyczką obsługującą \LaTeX a. Jest to standardowy edytor środowiska Gnome. Po instalacji wtyczki obsługującej \LaTeX~ zamienia się w wygodne i szybkie środowisko pracy.

\textbf{Dla testu łamania stron powtórzenia powyższego tekstu.}


Do przygotowania pliku źródłowego może zostać wykorzystany dowolny edytor tekstowy. Niektóre edytory, np. GEdit, mają wbudowane moduły ułatwiające składanie tekstów w LaTeXu (kolorowanie składni, skrypty kompilacji, itp.).
Jednym z bardziej znanych środowisk do składania dokumentów  \LaTeX a jest {\em TeXstudio}, oferujące kompletne środowisko pracy. Zobacz: \url{http://www.texstudio.org}
Bardzo dobrym środowiskiem jest również edytor gEdit z wtyczką obsługującą \LaTeX a. Jest to standardowy edytor środowiska Gnome. Po instalacji wtyczki obsługującej \LaTeX~ zamienia się w wygodne i szybkie środowisko pracy.
Po instalacji wtyczki obsługującej \LaTeX~ zamienia się w wygodne i szybkie środowisko pracy.

Do przygotowania pliku źródłowego może zostać wykorzystany dowolny edytor tekstowy. Niektóre edytory, np. GEdit, mają wbudowane moduły ułatwiające składanie tekstów w LaTeXu (kolorowanie składni, skrypty kompilacji, itp. itd. itp.).
Jednym z bardziej znanych środowisk do składania dokumentów  \LaTeX a jest {\em TeXstudio}, oferujące kompletne środowisko pracy. Zobacz: \url{http://www.texstudio.org}
Bardzo dobrym środowiskiem jest również edytor gEdit z wtyczką obsługującą \LaTeX a. Jest to standardowy edytor środowiska Gnome. Po instalacji wtyczki obsługującej \LaTeX~ zamienia się w wygodne i szybkie środowisko pracy.

Do przygotowania pliku źródłowego może zostać wykorzystany dowolny edytor tekstowy. Niektóre edytory, np. GEdit, mają wbudowane moduły ułatwiające składanie tekstów w LaTeXu (kolorowanie składni, skrypty kompilacji, itp.).
Jednym z bardziej znanych środowisk do składania dokumentów  \LaTeX a jest {\em TeXstudio}, oferujące kompletne środowisko pracy. Zobacz: \url{http://www.texstudio.org}
Bardzo dobrym środowiskiem jest również edytor gEdit z wtyczką obsługującą \LaTeX a. Jest to standardowy edytor środowiska Gnome. Po instalacji wtyczki obsługującej \LaTeX~ zamienia się w wygodne i szybkie środowisko pracy.

%---------------------------------------------------------------------------

\section{Przygotowanie dokumentu}
\label{sec:przygotowanieDokumentu}

Plik źródłowy \LaTeX a jest zwykłym plikiem tekstowym. Przygotowując plik
źródłowy warto wiedzieć o kilku szczegółach:

\begin{itemize}
\item
Poszczególne słowa oddzielamy spacjami, przy czym ilość spacji nie ma znaczenia.
Po kompilacji wielokrotne spacje i tak będą wyglądały jak pojedyncza spacja.
Aby uzyskać {\em twardą spację}, zamiast znaku spacji należy użyć znaku {\em
tyldy}.

\item
Znakiem końca akapitu jest pusta linia (ilość pusty linii nie ma znaczenia), a
nie znaki przejścia do nowej linii.

\item
\LaTeX~sam formatuje tekst. \textbf{Nie starajmy się go poprawiać}, chyba, że
naprawdę wiemy co robimy.
\end{itemize} 


