\chapter{Analysis and improvements of \textit{Rebvo} algorithm}
\label{cha:intro2}

This chapter contains thorough analysis of proposed algorithm. Firstly, changes to the original algorithm are listed. Then algorithm is carefully explained, step by step, using various examples. Chapter is concluded with miscellaneous remarks and experiment results.

\section{Changes to the algorithm}
\label{sec:changes}

TODO:

edge detection bucketing

median filter; regularize before and after Kalman

rho initialization (noise instead of ones)

small numerical improvements in LM


\section{Other considered changes}
\label{sec:rejected}

TODO:

scale from dimensions of known real-world objects, such as road signs or license plates

initialization from classical feature matching algorithm

keyline joining improvements: morphological operations, Y junctions

usage of corners

automatic DoG sigma choice


\section{Notation (Keyline structure)}

Pixels that contain subpixel edge positions are called Keylines and, after edge extraction, are stored as an array of structures defined in Table \ref{tab:keyline}.

\begin{table}[h]
	\centering
	
	\begin{threeparttable}
		\caption{Keyline structure}
		\label{tab:keyline}
		
		\begin{tabularx}{0.6\textwidth}{C{1} C{1}}
			\toprule
			\thead{Structure field\tnote{a}} & \thead{Description} \\
			\midrule
			$q$ & subpixel position in image \\
			$h$ & normalized $q$ (principal point moved to $(0,0)$ and coordinates divided by focal length $z_f$ \\
			$\rho$, $\sigma_{\rho}$ & estimated inverse depth: $\frac{1}{Z}$, and its variance \\
			$\rho_0$, $\sigma_{\rho_{0}}$ & inverse depth predicted by Kalman filter and its variance \\
			$g$ & edge gradient obtained from DoG \\
			$p_{id}$ & index of previous Keyline in an edge \\
			$n_{id}$ & index of next Keyline in an edge \\
			
			matching/forward/historia \\
			$h_0$ &  \\
			$g_{0}$ & gradient of matched Keyline from previous frame \\
			\bottomrule
		\end{tabularx}
		
		\begin{tablenotes}
			\footnotesize
			\item[a] Jakiś komentarz\textellipsis
		\end{tablenotes}
		
	\end{threeparttable}
\end{table}

% ---

\section{Edge extraction}

First main step of algorithm is edge extraction with subpixel accuracy. Extracted Keylines that are estimated to lie on same edge are joined.

\subsection{Zero crossing of DoG}
While many edge detection algorithms could be used in this step, authors of \textit{Rebvo} have chosen Difference of Gaussians, because it provides \cite{jose2015realtime}:
\begin{enumerate}
	\item repetivity -- an edge is detected similarly throughout consecutive frames,
	\item precision -- edge positions are accurate,
	\item low time \& memory complexity.
\end{enumerate}

Another advantage of DoG is that edge gradient can be obtained directly from normal vector of the fitted local plane.

TODO:
frame for rectification leftovers removal

DOG is performed + example

for every px minus windows:

test 1: check if gradient if sufficient (todo check if most px are discarded)

test2: check if plues and minues are balanced, 1px edge

fitting to plane, check if fitting was ok

test 3: zero crossing; if lies inside the pixel

test on dog: check if edge gradient is strong enough

image: quiver

todo: prove that noise for rho is better than ones

imgmask with KL indices

\subsection{Keyline joining}

After obtaining individual Keylines, they are joined together to form connected edges.
For each Keyline, its neighbors are searched among 8 bordering pixels in directions perpendicular to edge gradient.
In most of its steps, the algorithm considers only individual Keylines. Their neighbors are used only in Matching step to ... % todo
and in Regularization step, when ... % todo
is averaged over Keyline and its neighbors.

todo: image of joined edges; pruning of edge ends and 3-px edges are removed


%--

\section{Edge tracking}

During minimization, previous and current Keylines are initially@ matched

% minimizer
% auxiliary image: a lookup table for minimizer
% estimate quantile - its purpose is to cutoff KLs that have too high uncertainty with respect to all KLs
% project KL_prev to 3D using previously estimated rhos
% one iteration of LM:
%     using jacobian, new state vector is calculated for next iteration

%     apply transformation given in argument
%     project previous points from these 3D positions to 2D
%     for every previous KL perfoms test to check if it should take part in minimization:
%        test1: uncertainty below previously estimated quantile
%		 test2: (excluding minimalization for very first frame) check if KL has appeared before
%	     estimate rewieghting
%		 test3: check if projected KL lies within image frame
%        test4: check if there is a KL in this pixel using auxiliary image
%        test5: compare their gradients (escobar)
%        we get residual (DResidual) & weighted residual (fm) projected in direction of gradient
%      calculate jacobians using fishy equations
%      if there was gain, use these new parameters, otherwise ...
% double init(zeros and priors): 3 iterations with no reweighting
% use better result, proceed for 15 iterations with reweighting
% transformation that resulted in lowest score is considered the optimal transformation between frames
% estimate uncertainty as inverse of final jacobian

\subsection{Warping function}

\subsection{Auxiliary image}

todo: a lookup table for minimizer

\subsection{Keyline matching criteria}

% todo Jan: figure

\subsection{Energy minimization}

\subsection{Initial conditions}

\subsection{Reweighing}

%--

\section{Mapping}

% forward rotate
% KL pixel positions, represented in normalized homo coords, are rotated using obtained rotation
% if they don't go to infinity, they are projected back onto image plane
% & nasty gradient rotation

\subsection{Forward matching}

% forward match
% using information about which KLs were matched to which, following fields are propagated: rho, s_rho, gradient, history and index of the matched KL
% todo: could new edges apprer without directed match

\subsection{Directed matching}

%directed match
% for each KL:
%   backrotate KL and cast it onto old edge map mask
%   using this position and backrotated velocity, obtain halfline in which this KL could have been moved (a halfline, because we only know that its rho is positive) - estimated pixel displacement
%   estimate uncertainty using projected position and Rvel
%   caonstrain displacement search radius & define starting point
% todo: check if perpedicular displacement if really common
%
% search from starting point alternating between one direction and the other
% check if this pixel there is a KL to match to
% perform 3 tests to filter outliers + image (eg. 3 consecutive images where an outlier disappears):
%  gradient angle similarity
%  gradient modulus similarity
%  motion consistency

% if there aren't enough matched KLs (500), reset

\subsection{Regularization}

% optional regularization, performed twice
% main assumption is that KLs neighboring on an edge ale located near each other in 3D, so their depth should be similiar
% image: show that this isn't ALWAYS true
% for each KL:
%  check if 2 neighbors pass test:
%   if depths outweight uncertainties (probabilistic uncertainty)
%   if angle between gradients is below threshold
%  if tests are passed, rho and s_rho are smoothened taking weighted mean of its value and its neighbors

\subsection{Depth estimation}

% kalman fiter
% kill me pls
% inverse depth is constrained between certain min and max

\subsection{Scale correction}


% optional estimate rescaling
% according to Tarrio et al, EKF is biased in rho estimation, so a global "shrinking factor" can be applied to depths and unvertainties.


% ----------------------







