\chapter{Theoretical background and state-of-the-art}
\label{cha:intro}


In this chapter theory behind computer vision, 3D reconstruction and mathematical apparatus used in our algorithm are introduced@. Following topics are briefly reviewed@: pinhole camera model, basics of stereo vision, egomotion and 3D image transformations. Difference of Gaussians edge detection, Levenberg-Marquadt algorithm and Kalman filter can be found in Appendices B, C and D, respectively.

Literature connected with this topic is reviewed@. Finally, thorough analysis of Rebvo algorithm \cite{jose2015realtime} is performed@.

%-----------------

\section{Pinhole camera model}
\label{sec:pinhole}


Pinhole camera model is a widely used and simple model that establishes connection between world coordinates $\Re^3$ and image-domain coordinates $\Re^2$ \cite{hartley2003multiple} (i.e. projective geometry).

Figure. Every 3D point $Q$ can be associated with a ray $QO$ that passes through camera origin $O$, usually defined as origin of the 3D coordinate system origin. Such ray can be defined with homogeneous coordinates as set of points $(X, Y, Z)$ that satisfy Eq.~\ref{eq:homo}

\begin{equation}
(X, Y, Z) = k(Q_x, Q_y, Q_z)
\label{eq:homo}
\end{equation}
where:
\begin{eqwhere}[2cm]
	\item[$k$] real parameter, \(k \neq 0\)
	\item[$Q$] world coordinates point
\end{eqwhere}

Image plane \(\pi\) is a rectangle parallel to plane \(XOY\). Its distance from origin is equal to \(f\) (focal length). Usually it is assumed that image plane's \(z\) coordinate is positive -- otherwise formed image would be upside down. Point of \(OZ\) and \(\pi\) intersection is called principal point.

World coordinate points Q are projected onto \(\pi\) as \(Q'\), thus forming a 2D image.

%---------

\section{Stereo vision}
\label{sec:stereo}


%---------

\section{Egomotion}
\label{sec:ego}

%-----------

\section{3D image transformations}
\label{sec:3dtrans}

\subsection{Notations of 3D rotation}

Rodriguez, quaternion, Euler, expm

\subsection{3D translation and rotation}

% -----------

\section{Literature}


% -------------

\section{The \textit{Rebvo} algorithm}

In this section the \textit{Rebvo} algorithm \cite{jose2015realtime} description is paraphrased for need of this thesis.

The algorithm is similar to semi-dense methods that achieve SLAM only with extracted features (in this case - edges). It is not a full SLAM system, so only 2 consecutive frames are stored at each time and no global map is created. Information from previous frames is retained as estimated depth.

% ---

\subsection{Notation (Keyline structure)}

Pixels that contains subpixel edge positions are called Keylines and, after edge extraction, are stored as an array of structures defined in Table \ref{tab:keyline}.

\begin{table}[h]
	\centering
	
	\begin{threeparttable}
		\caption{Keyline structure}
		\label{tab:keyline}
		
		\begin{tabularx}{0.6\textwidth}{C{1}}
			\toprule
			\thead{Nagłówek\tnote{a}} \\
			\midrule
			Tekst 1 \\
			Tekst 2 \\
			\bottomrule
		\end{tabularx}
		
		\begin{tablenotes}
			\footnotesize
			\item[a] Jakiś komentarz\textellipsis
		\end{tablenotes}
		
	\end{threeparttable}
\end{table}

% ---

\subsection{Edge extraction}

First main step of algorithm is edge extraction with subpixel accuracy. Extracted Keylines that are estimated to lie on same edge are joined.

\subsubsection{Zero crossing of DoG}
While many edge detection algorithms could be used in this step, authors of \textit{Rebvo} have chosen Difference of Gaussians, because it provides \cite{jose2015realtime}:
\begin{enumerate}
	\item repetivity -- an edge is detected similarly throughout consecutive frames,
	\item precision -- edge positions are accurate,
	\item low time \& memory complexity.
\end{enumerate}

Another advantage of DoG is that edge gradient can be obtained directly from normal vector of the fitted local plane.

\subsubsection{Keyline joining}

After obtaining individual Keylines, they are joined together to form continuous (from consecutive pixels) edges. % todo footnote
For each Keyline, its neighbors are searched among 8 bordering pixels in direction perpendicular to edge gradient.
Most of the time, the algorithm considers only individual Keylines. Their neighbors are used only in Matching step to ... % todo
and in Regularization step, when ... % todo
is averaged over Keyline and its neighbors.


%--

\subsection{Edge tracking}

During minimization, previous and current Keylines are initially@ matched

\subsubsection{Warping function}

\subsubsection{Auxiliary image}

\subsubsection{Keyline matching criteria}

% todo Jan: figure

\subsubsection{Energy minimization}

\subsubsection{Initial conditions}

\subsubsection{Reweighing}

%--

\subsection{Mapping}

\subsubsection{Forward matching}

\subsubsection{Directed matching}

\subsubsection{Regularization}

\subsubsection{Depth estimation}

\subsubsection{Scale correction}